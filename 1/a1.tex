\documentclass[12pt]{article}
\usepackage{amsmath}
\usepackage{latexsym}
\usepackage{amsfonts}
\usepackage[normalem]{ulem}
\usepackage{soul}
\usepackage{array}
\usepackage{amssymb}
\usepackage{extarrows}
\usepackage{graphicx}
\usepackage[backend=biber,
style=numeric,
sorting=none,
isbn=false,
doi=false,
url=false,
]{biblatex}\addbibresource{bibliography.bib}

\usepackage{subfig}
\usepackage{wrapfig}
\usepackage{wasysym}
\usepackage{enumitem}
\usepackage{adjustbox}
\usepackage{ragged2e}
\usepackage[svgnames,table]{xcolor}
\usepackage{tikz}
\usepackage{longtable}
\usepackage{changepage}
\usepackage{setspace}
\usepackage{hhline}
\usepackage{multicol}
\usepackage{tabto}
\usepackage{float}
\usepackage{multirow}
\usepackage{makecell}
\usepackage{fancyhdr}
\usepackage[toc,page]{appendix}
\usepackage[hidelinks]{hyperref}
\usetikzlibrary{shapes.symbols,shapes.geometric,shadows,arrows.meta}
\tikzset{>={Latex[width=1.5mm,length=2mm]}}
\usepackage{flowchart}\usepackage[paperheight=11.69in,paperwidth=8.27in,left=1.25in,right=1.25in,top=1.0in,bottom=1.0in,headheight=1in]{geometry}
\usepackage[utf8]{inputenc}
\usepackage[T1]{fontenc}
\TabPositions{0.29in,0.58in,0.87in,1.16in,1.45in,1.74in,2.03in,2.32in,2.61in,2.9in,3.19in,3.48in,3.77in,4.06in,4.35in,4.64in,4.93in,5.22in,5.51in,}

\urlstyle{same}

\renewcommand{\_}{\kern-1.5pt\textunderscore\kern-1.5pt}



\setcounter{tocdepth}{5}
\setcounter{secnumdepth}{5}



\setlistdepth{9}
\renewlist{enumerate}{enumerate}{9}
		\setlist[enumerate,1]{label=\arabic*)}
		\setlist[enumerate,2]{label=\alph*)}
		\setlist[enumerate,3]{label=(\roman*)}
		\setlist[enumerate,4]{label=(\arabic*)}
		\setlist[enumerate,5]{label=(\Alph*)}
		\setlist[enumerate,6]{label=(\Roman*)}
		\setlist[enumerate,7]{label=\arabic*}
		\setlist[enumerate,8]{label=\alph*}
		\setlist[enumerate,9]{label=\roman*}

\renewlist{itemize}{itemize}{9}
		\setlist[itemize]{label=$\cdot$}
		\setlist[itemize,1]{label=\textbullet}
		\setlist[itemize,2]{label=$\circ$}
		\setlist[itemize,3]{label=$\ast$}
		\setlist[itemize,4]{label=$\dagger$}
		\setlist[itemize,5]{label=$\triangleright$}
		\setlist[itemize,6]{label=$\bigstar$}
		\setlist[itemize,7]{label=$\blacklozenge$}
		\setlist[itemize,8]{label=$\prime$}

\setlength{\topsep}{0pt}\setlength{\parindent}{0pt}
\renewcommand{\arraystretch}{1.3}


%%%%%%%%%%%%%%%%%%%% Document code starts here %%%%%%%%%%%%%%%%%%%%



\begin{document}
\begin{Center}
{\fontsize{16pt}{19.2pt}\selectfont \textbf{CSC236 Assignment 1 }\par}
\end{Center}\par

\begin{Center}
{\fontsize{16pt}{19.2pt}\selectfont \textbf{Jiahong Zhai (zhaijia3)}\par}
\end{Center}\par

\begin{Center}
{\fontsize{16pt}{19.2pt}\selectfont \textbf{Due Date: 2020.1.30}\par}
\end{Center}\par


\vspace{\baselineskip}
\begin{adjustwidth}{0.25in}{0.0in}
{\fontsize{14pt}{16.8pt}\selectfont P(n): f(n) = 2\textsuperscript{(n/2)} ⋅ ( \( \frac{n}{2} \) )!\par}\par

\end{adjustwidth}

\begin{adjustwidth}{0.25in}{0.0in}
{\fontsize{14pt}{16.8pt}\selectfont WTS: $ \forall $  n $ \in $  ℕ, n is even $ \Rightarrow $  P(n)\par}\par

\end{adjustwidth}

\begin{adjustwidth}{0.25in}{0.0in}
{\fontsize{14pt}{16.8pt}\selectfont Base Case:\par}\par

\end{adjustwidth}

\begin{adjustwidth}{0.25in}{0.0in}
\tab \tab {\fontsize{14pt}{16.8pt}\selectfont f(0) = 1 = (2\textsuperscript{(0/2)}) ⋅ ( \( \frac{0}{2} \) )!\par}\par

\end{adjustwidth}

\begin{adjustwidth}{0.25in}{0.0in}
\tab \tab {\fontsize{14pt}{16.8pt}\selectfont P(0) holds\par}\par

\end{adjustwidth}

{\fontsize{14pt}{16.8pt}\selectfont Inductive Steps:\par}\par

\begin{adjustwidth}{0.25in}{0.0in}
\tab {\fontsize{14pt}{16.8pt}\selectfont \tab Let n be an arbitrary even natural number\par}\par

\end{adjustwidth}

\begin{adjustwidth}{0.25in}{0.0in}
\tab {\fontsize{14pt}{16.8pt}\selectfont \tab Assume P(n) holds:\par}\par

\end{adjustwidth}

\begin{adjustwidth}{0.25in}{0.0in}
\tab \tab {\fontsize{14pt}{16.8pt}\selectfont \tab f(n) = 2\textsuperscript{(n/2)} ⋅ ( \( \frac{n}{2} \) )!\par}\par

\end{adjustwidth}

\begin{adjustwidth}{0.25in}{0.0in}
\tab {\fontsize{14pt}{16.8pt}\selectfont \tab WTS: P(n+2)\tab \tab i.e., f(n+2) = 2\textsuperscript{((n+2)/2)} ⋅ ( \( \frac{n+2}{2} \) )!\par}\par

\end{adjustwidth}

\begin{adjustwidth}{0.25in}{0.0in}
\tab \tab \tab {\fontsize{14pt}{16.8pt}\selectfont f(n + 2)  =\  (n + 2) ⋅ f(n)\par}\par

\end{adjustwidth}

\begin{adjustwidth}{0.25in}{0.0in}
\tab \tab \tab \tab {\fontsize{14pt}{16.8pt}\selectfont \tab \   =\  (n + 2) ⋅ 2\textsuperscript{(n/2)} ⋅ ( \( \frac{n}{2} \) )!\par}\par

\end{adjustwidth}

\begin{adjustwidth}{0.25in}{0.0in}
\tab \tab \tab \tab {\fontsize{14pt}{16.8pt}\selectfont \tab \   =\  ( \( \frac{n}{2} \)  + 1) ⋅ 2 ⋅ 2\textsuperscript{(n/2)} ⋅ ( \( \frac{n}{2} \) )!\par}\par

\end{adjustwidth}

\begin{adjustwidth}{0.25in}{0.0in}
\tab \tab \tab \tab {\fontsize{14pt}{16.8pt}\selectfont \tab \   =\  2 ⋅ 2\textsuperscript{(n/2)} ⋅ ( \( \frac{n}{2} \)  + 1) ⋅ ( \( \frac{n}{2} \) )!\par}\par

\end{adjustwidth}

\begin{adjustwidth}{0.25in}{0.0in}
\tab \tab \tab \tab {\fontsize{14pt}{16.8pt}\selectfont \tab \   =\  2 ⋅ 2\textsuperscript{(n/2)} ⋅ ( \( \frac{n}{2} \)  + 1)!\par}\par

\end{adjustwidth}

\begin{adjustwidth}{0.25in}{0.0in}
\tab \tab \tab \tab {\fontsize{14pt}{16.8pt}\selectfont \tab \ \  =\  2\textsuperscript{((n+2)/2)} ⋅ ( \( \frac{n + 2}{2} \) )!\par}\par

\end{adjustwidth}

\begin{adjustwidth}{0.25in}{0.0in}
\tab {\fontsize{14pt}{16.8pt}\selectfont \tab So P(n) $ \Rightarrow $  P(n + 2) for all even natural number n\par}\par

\end{adjustwidth}

\begin{adjustwidth}{0.25in}{0.0in}
\tab {\fontsize{14pt}{16.8pt}\selectfont \tab So, $ \forall $  n $ \in $  ℕ, n is even $ \Rightarrow $  P(n)\tab \tab \tab \tab \tab \tab \tab \tab \tab $\centerdot$ \par}\par

\end{adjustwidth}

\begin{enumerate}
	\item {\fontsize{14pt}{16.8pt}\selectfont (a)\par}\par

{\fontsize{14pt}{16.8pt}\selectfont WTS: $ \forall $  n $ \in $  ℕ, P(n)\par}\par

{\fontsize{14pt}{16.8pt}\selectfont Let n $ \in $  ℕ\par}\par

{\fontsize{14pt}{16.8pt}\selectfont Inductive Hypothesis: If n > 1, P(k) holds for $ \forall $  k $ \in $  ℕ, 0 < k < n\par}\par

{\fontsize{14pt}{16.8pt}\selectfont Case 1: n = 1\par}\par

\tab \tab {\fontsize{14pt}{16.8pt}\selectfont P(1) holds\tab $\#$  (1)\par}\par

{\fontsize{14pt}{16.8pt}\selectfont Case 2: n > 1, n is even\par}\par

\tab \tab {\fontsize{14pt}{16.8pt}\selectfont P( \( \frac{n}{2} \) ) holds\tab \tab \tab \tab $\#$   \( \frac{n}{2}~ \) $ \in $  ℕ $\wedge$  I.H\par}\par

\tab \tab {\fontsize{14pt}{16.8pt}\selectfont P( \( \frac{n}{2} \) ) $ \Rightarrow $  P(n)\tab \tab \tab $\#$  (3)\par}\par

\tab {\fontsize{14pt}{16.8pt}\selectfont \tab P(n) holds\par}\par

{\fontsize{14pt}{16.8pt}\selectfont Case 3: n > 1, n is odd\par}\par

\tab \tab {\fontsize{14pt}{16.8pt}\selectfont P( \( \frac{n + 1}{2} \) ) holds\tab \tab \tab $\#$   \( \frac{n + 1}{2}~ \) $ \in $  ℕ $\wedge$   \( \frac{n+1}{2} \)  < n (Since n < 1) $\wedge$  I.H\par}\par

\tab \tab {\fontsize{14pt}{16.8pt}\selectfont P( \( \frac{n + 1}{2} \) ) $ \Rightarrow $  P(n + 1)\tab \tab $\#$  (3)\par}\par

\tab \tab {\fontsize{14pt}{16.8pt}\selectfont P(n + 1) $ \Rightarrow $  P(n)\tab \tab \tab $\#$  (2)\par}\par

\tab {\fontsize{14pt}{16.8pt}\selectfont \tab P(n) holds\par}\par

{\fontsize{14pt}{16.8pt}\selectfont Case 4: n = 0\par}\par

\tab \tab {\fontsize{14pt}{16.8pt}\selectfont P(1) holds\tab \tab \tab \tab $\#$  (1)\par}\par

\tab \tab {\fontsize{14pt}{16.8pt}\selectfont P(1) $ \Rightarrow $  P(0)\tab \tab \tab \tab $\#$  (2)\par}\par

\tab \tab {\fontsize{14pt}{16.8pt}\selectfont P(0) holds\par}\par


\vspace{\baselineskip}
{\fontsize{14pt}{16.8pt}\selectfont $ \forall $  n $ \in $  ℕ, P(n) holds\par}\par

{\fontsize{14pt}{16.8pt}\selectfont $\centerdot$ \par}\par

\begin{enumerate}
	\item {\fontsize{14pt}{16.8pt}\selectfont (b)\par}\par

{\fontsize{14pt}{16.8pt}\selectfont We can only prove P(1) and P(0)\par}\par

{\fontsize{14pt}{16.8pt}\selectfont P(1) holds\tab \tab $\#$  (1)\par}\par

{\fontsize{14pt}{16.8pt}\selectfont P(1) $ \Rightarrow $  P(0)\tab $\#$  (2) $\wedge$  1 $ \in $  ℕ\textsuperscript{+}\par}\par

{\fontsize{14pt}{16.8pt}\selectfont We can’t go any further\par}\par

{\fontsize{14pt}{16.8pt}\selectfont $\centerdot$ \par}\par


\vspace{\baselineskip}

\vspace{\baselineskip}

\vspace{\baselineskip}
{\fontsize{14pt}{16.8pt}\selectfont Let S be such a set defined in the problem.\par}\par

{\fontsize{14pt}{16.8pt}\selectfont P(n): n does not contain substring yh.\par}\par

{\fontsize{14pt}{16.8pt}\selectfont Let n $ \in $  S\par}\par

{\fontsize{14pt}{16.8pt}\selectfont WTS $ \forall $  n $ \in $  S, P(n)\par}\par

{\fontsize{14pt}{16.8pt}\selectfont Base Case: n = u\par}\par

\tab {\fontsize{14pt}{16.8pt}\selectfont \tab u $ \in $  S\par}\par

\tab {\fontsize{14pt}{16.8pt}\selectfont \tab u does not contain yh\par}\par

\tab {\fontsize{14pt}{16.8pt}\selectfont \tab P(u) holds\par}\par

{\fontsize{14pt}{16.8pt}\selectfont Inductive steps:\par}\par

\tab {\fontsize{14pt}{16.8pt}\selectfont \tab Let s1, s2 $ \in $  S\par}\par

\tab {\fontsize{14pt}{16.8pt}\selectfont \tab Inductive Hypothesis: Assume P(s1) $\wedge$  P(s2)\par}\par

\tab {\fontsize{14pt}{16.8pt}\selectfont \tab Lemma 1: y can’t be postfix of any element in S\par}\par

{\fontsize{14pt}{16.8pt}\selectfont Because y can only add to one string’s leftmost digit to form a new element in S, and y is not in S\par}\par

\tab {\fontsize{14pt}{16.8pt}\selectfont \tab Lemma 2: h can’t be prefix of any element in S\par}\par

{\fontsize{14pt}{16.8pt}\selectfont Because h can only add to one string’s rightmost digit to form a new element in S, and h is not in S\par}\par

\tab {\fontsize{14pt}{16.8pt}\selectfont \tab Case 1: n = ys1\par}\par

\tab \tab \tab {\fontsize{14pt}{16.8pt}\selectfont \tab The first digit in s1 is not h\tab $\#$  Lemma 2\par}\par

\tab \tab \tab {\fontsize{14pt}{16.8pt}\selectfont \tab So, the first two digits in ys1 is not yh.\par}\par

\tab \tab \tab {\fontsize{14pt}{16.8pt}\selectfont \tab So, ys1 does not contain yh.\par}\par

\tab \tab \tab {\fontsize{14pt}{16.8pt}\selectfont \tab P(ys1) holds\par}\par

\tab {\fontsize{14pt}{16.8pt}\selectfont \tab Case 2: n = s1h\par}\par

\tab \tab \tab {\fontsize{14pt}{16.8pt}\selectfont \tab The last digit in s1 is not y\tab $\#$  Lemma 1\par}\par

\tab \tab \tab {\fontsize{14pt}{16.8pt}\selectfont \tab So, the last two digits in s1h is not yh.\par}\par

\tab \tab \tab {\fontsize{14pt}{16.8pt}\selectfont \tab So, s1h does not contain yh.\par}\par

\tab \tab \tab {\fontsize{14pt}{16.8pt}\selectfont \tab P(s1h) holds\par}\par

\tab {\fontsize{14pt}{16.8pt}\selectfont \tab Case 3: n = s1s2\par}\par

\tab \tab \tab {\fontsize{14pt}{16.8pt}\selectfont \tab The last digit in s1 is not y\tab $\#$  Lemma 1\par}\par

\tab \tab \tab {\fontsize{14pt}{16.8pt}\selectfont \tab The first digit in s2 is not h\tab $\#$  Lemma 2\par}\par

\tab \tab \tab {\fontsize{14pt}{16.8pt}\selectfont \tab So, s1s2 does not contain yh.\par}\par

\tab \tab \tab {\fontsize{14pt}{16.8pt}\selectfont \tab P(s1s2) holds\par}\par

{\fontsize{14pt}{16.8pt}\selectfont \tab So, $ \forall $  n $ \in $  S, P(n)\par}\par

\begin{FlushRight}
{\fontsize{14pt}{16.8pt}\selectfont $\centerdot$ \par}
\end{FlushRight}\par


\vspace{\baselineskip}

\vspace{\baselineskip}
\begin{FlushLeft}
{\fontsize{14pt}{16.8pt}\selectfont I will prove this problem by two steps. \par}
\end{FlushLeft}\par


\vspace{\baselineskip}
\begin{FlushLeft}
{\fontsize{14pt}{16.8pt}\selectfont First, I will prove that for any natural number n, there exists at least one natural number m, containing exactly n sub-strings in its decimal representation which are prime numbers.\par}
\end{FlushLeft}\par


\vspace{\baselineskip}
\begin{FlushLeft}
{\fontsize{14pt}{16.8pt}\selectfont Second, I will prove that for any natural number n, there exists a smallest natural number A(n), containing exactly n sub-strings in its decimal representation which are prime numbers.\par}
\end{FlushLeft}\par


\vspace{\baselineskip}
\begin{FlushLeft}
{\fontsize{14pt}{16.8pt}\selectfont Step 1:\par}
\end{FlushLeft}\par

\begin{FlushLeft}
{\fontsize{14pt}{16.8pt}\selectfont P(n): $ \exists $  m $ \in $  ℕ, m containing exactly n sub-strings in its decimal representation which are prime numbers.\par}
\end{FlushLeft}\par

\begin{FlushLeft}
{\fontsize{14pt}{16.8pt}\selectfont WTS: $ \forall $  n $ \in $  ℕ, P(n)\par}
\end{FlushLeft}\par

\begin{FlushLeft}
{\fontsize{14pt}{16.8pt}\selectfont Let n $ \in $  ℕ\par}
\end{FlushLeft}\par

\begin{FlushLeft}
{\fontsize{14pt}{16.8pt}\selectfont Case 1: n = 0\par}
\end{FlushLeft}\par

\begin{FlushLeft}
{\fontsize{14pt}{16.8pt}\selectfont \tab Let m be 8\par}
\end{FlushLeft}\par

\begin{FlushLeft}
{\fontsize{14pt}{16.8pt}\selectfont 8 contains exactly 0 sub-strings in its decimal representation which are prime numbers.\par}
\end{FlushLeft}\par

\begin{FlushLeft}
{\fontsize{14pt}{16.8pt}\selectfont P(0) holds.\par}
\end{FlushLeft}\par


\vspace{\baselineskip}
\begin{FlushLeft}
\tab {\fontsize{14pt}{16.8pt}\selectfont \tab Case 2: n > 0\par}
\end{FlushLeft}\par

\begin{FlushLeft}
\tab \tab {\fontsize{14pt}{16.8pt}\selectfont \tab Let m be 555555$ \ldots $ $ \ldots $ 5555555 (n digit of 5)\par}
\end{FlushLeft}\par

\begin{FlushLeft}
\tab \tab {\fontsize{14pt}{16.8pt}\selectfont \tab 5 is a prime number\par}
\end{FlushLeft}\par

\begin{FlushLeft}
{\fontsize{14pt}{16.8pt}\selectfont 55$ \ldots $ $ \ldots $ 55(k digit of 5, k $ \in $  ℕ, k > 1) is not a prime number, because it can be divided by 5. Proof below:\par}
\end{FlushLeft}\par

\begin{FlushLeft}
{\fontsize{14pt}{16.8pt}\selectfont $\#$  55$ \ldots $ $ \ldots $ 55 = 55$ \ldots $ $ \ldots $ 50 + 5 = 10t + 5(let t be such a natural number). WTS: $ \exists $  r $ \in $  ℕ, 10t + 5 = 5r.\par}
\end{FlushLeft}\par

\begin{FlushLeft}
{\fontsize{14pt}{16.8pt}\selectfont Let r = 2t + 1\par}
\end{FlushLeft}\par

\begin{FlushLeft}
{\fontsize{14pt}{16.8pt}\selectfont 5r = 10t + 5 = 55$ \ldots $ $ \ldots $ 55 \par}
\end{FlushLeft}\par

\begin{FlushLeft}
{\fontsize{14pt}{16.8pt}\selectfont 555555$ \ldots $ $ \ldots $ 5555555 (n digit of 5) contains exactly n sub-strings in its decimal representation which are prime numbers.\par}
\end{FlushLeft}\par

\begin{FlushLeft}
{\fontsize{14pt}{16.8pt}\selectfont P(n) holds.\par}
\end{FlushLeft}\par

\begin{FlushLeft}
\tab {\fontsize{14pt}{16.8pt}\selectfont \tab So, $ \forall $  n $ \in $  ℕ, P(n)\par}
\end{FlushLeft}\par

\tab 
\vspace{\baselineskip}\begin{FlushLeft}
{\fontsize{14pt}{16.8pt}\selectfont \tab Step 2: \par}
\end{FlushLeft}\par

\begin{FlushLeft}
{\fontsize{14pt}{16.8pt}\selectfont Q(n): $ \exists $  A(n) $ \in $  ℕ, A(n) is the smallest number containing exactly n sub-strings in its decimal representation which are prime numbers.\par}
\end{FlushLeft}\par

\begin{FlushLeft}
{\fontsize{14pt}{16.8pt}\selectfont WTS: $ \forall $  n $ \in $  ℕ, Q(n)\par}
\end{FlushLeft}\par

\begin{FlushLeft}
{\fontsize{14pt}{16.8pt}\selectfont Let n $ \in $  ℕ\par}
\end{FlushLeft}\par


\vspace{\baselineskip}
\begin{FlushLeft}
{\fontsize{14pt}{16.8pt}\selectfont Let S(n) be the set of all natural numbers containing exactly n sub-strings in its decimal representation which are prime numbers.\par}
\end{FlushLeft}\par

\begin{FlushLeft}
{\fontsize{14pt}{16.8pt}\selectfont From step 1, it is proven that S(n) is not empty. And it’s obvious that S(n) is a subset of ℕ.\par}
\end{FlushLeft}\par

\begin{FlushLeft}
{\fontsize{14pt}{16.8pt}\selectfont $ \exists $  j $ \in $  S(n), $ \forall $  i $ \in $  S(n), i $ \geq $  j\tab \tab $\#$ Well-Ordering Principle\par}
\end{FlushLeft}\par

\begin{FlushLeft}
{\fontsize{14pt}{16.8pt}\selectfont Let A(n) = j\par}
\end{FlushLeft}\par

\begin{FlushLeft}
{\fontsize{14pt}{16.8pt}\selectfont A(n) is the smallest number containing exactly n sub-strings in its decimal representation which are prime numbers.\par}
\end{FlushLeft}\par

\begin{FlushRight}
{\fontsize{14pt}{16.8pt}\selectfont $\centerdot$ \par}
\end{FlushRight}\par


\end{enumerate}
\printbibliography
\end{document}